\section{Sposoby dojścia do celu}




\subsection{Narzędzia}

% Symbole do głębokości=3 w itemize zmienione na kropki
\renewcommand\labelitemii{$\bullet$}
\renewcommand\labelitemiii{$\bullet$}

\begin{itemize}
    \item język Python>=3.7.
    \begin{itemize}
        \item Narzędzia wykorzystane do przetwarzania obrazów:
        \begin{itemize}     
            \item \href{https://pillow.readthedocs.io/en/stable/}{Pillow - zbiór narzędzi do obróbki grafiki}\cite{pillow},
            \item \href{https://numpy.org/}{numpy - obliczenia naukowe, pythonowa alternatywa dla języka MATLAB}\cite{numpy},
            \item \href{https://pypi.org/project/opencv-python/}{opencv-python - \say{Open Source Computer Vision Library
            }}\cite{opencv-python},
            \item \href{http://dlib.net/}{dlib - zbiór narzędzi uczenia maszynowego}\cite{dlib}.
        \end{itemize}   
        \item Inne narzędzia:
        \begin{itemize}
            \item \href{https://pypi.org/project/PyQt5/}{PyQt5 - framework GUI}\cite{pyqt5}.
        \end{itemize}
    \end{itemize}
\end{itemize}

\subsection{Algorytmy}
\begin{itemize}
    \item Algorytm \href{http://www.csc.kth.se/~vahidk/face_ert.html}{\say{Ensemble of Regression Trees (ERT)}}\cite{face-ert}.
\end{itemize}

\subsection{Dane}
\begin{itemize}
\item Zestaw danych \href{https://ibug.doc.ic.ac.uk/resources/300-W/}{iBUG 300-W}\cite{300-w}. 
\end{itemize}


\subsection{Plan realizacji}
\begin{itemize}

    \item Rozpoznawanie twarzy: 
    \begin{enumerate}
        \item określenie potrzebnych punktów charakterystycznych.
        \item Wytrenowanie modelu predyktora na podstawie zestawu danych iBUG 300-W\cite{300-w}, mając na uwadze szybkość predykcji, potrzebną do rysowania masek w czasie rzeczywistym. 
        \item Zastosowanie modelu wynikowego do predykcji położenia punktów charakterystycznych na obrazie pobieranym ze źródła obrazu (docelowo kamery).
    \end{enumerate}
  
    \item Maski:
    \begin{enumerate}
        \item Określenie interfejsów wejścia/wyjścia generycznej maski.
        \item Określenie wymagań co do rodzajów masek, w tym aspektów estetycznych.
        \item Implementacja rysowania konkretnych masek, zgodnych z ustalonym interfejsem.
    \end{enumerate}
  
    \item Interfejs graficzny:
    \begin{enumerate}
        \item określenie wymagań - funkcjonalności interfejsu graficznego oraz jego widoków,
        \item projekt położenia elementów typu \say{wireframe}, dla poszczególnych widoków.
        \item Implementacja widoków.
        \item Integracja GUI z logiką programu.
    \end{enumerate}

\end{itemize} 
