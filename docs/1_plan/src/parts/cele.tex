\section{Cele projektu}
W trakcie realizowania projektu naszym głównym celem było poznanie technik i nardzędzi do rozpoznawania twarzy
oraz śledzenia jej w czasie rzeczywistym. To automatycznie narzuciło wymóg sprawnego i świadomego korzystania 
z zewnętrznych bibliotek oraz tworzenia i trenowania swoich własnych modeli sztucznej inteligencji. 
Poza aspektem edukacyjnym i chęcią poszerzania swoich horyzontów, bardzo zależy nam na immersywności aplikacji i możliwości
późniejszego rozwoju i skalowania produktu końcowego.
Ostatecznie aplikacja ma być naszą interpretacją ogólnodostępnych, komercyjnych nardzędzi (takich jak Instagram lub Snapchat)
umożliwiających interakcję z użytkownikiem poprzez nakładanie filtrów lub masek na twarz.

